

%%%%%%%%%%%%%%%%%%%%%%%%%%%%%%%%%%%%%%%%%%%%%%%%%%%%%%%%%%%%%%%%%%%%%%%%%%%
%%%%%%%%%%%%%%%%%%%%%%%%%%%%%%%%%%%%%%%%%%%%%%%%%%%%%%%%%%%%%%%%%%%%%%%%%%%


\documentclass[12pt]{article}


\usepackage{amsmath, amsthm, amssymb, setspace, fullpage, apacite, enumitem, listings}
\usepackage[english]{babel}


\renewcommand{\qedsymbol}{$\scriptstyle \blacksquare$}
\renewcommand{\vec}[1]{\mbox{\boldmath$#1$}}
\newcommand{\dto}{\overset{d}{\to}}
\newcommand{\pto}{\overset{p}{\to}}
\newcommand{\E}{\mathrm{E}}
\newcommand{\F}{\mathfrak{F}}
\newcommand{\I}{\mathrm{I}}
\newcommand{\M}{\mathfrak{M}}
\newcommand{\N}{\mathrm{N}}
\newcommand{\diag}{\mathrm{diag}}
\renewcommand{\P}{\mathrm{P}}
\newcommand{\Q}{\mathrm{Q}}
\newcommand{\Cov}{\mathrm{Cov}}
\newcommand{\Var}{\mathrm{Var}}
\newcommand{\betav}{\vec{\beta}}
\newcommand{\betahat}{\hat{\vec{\beta}}}
\newcommand{\argmax}{\operatornamewithlimits{argmax}}
\newcommand{\argmin}{\operatornamewithlimits{argmin}}
\newcommand{\plim}{\operatornamewithlimits{plim}}
\newcommand{\interior}{\operatornamewithlimits{int}}

\newcommand\independent{\protect\mathpalette{\protect\independenT}{\perp}}
\def\independenT#1#2{\mathrel{\setbox0\hbox{$#1#2$}%
\copy0\kern-\wd0\mkern4mu\box0}}  % statistical independence symbol

\newtheorem{thm}{Theorem}[section]
\newtheorem{corollary}[thm]{Corollary}
\newtheorem{lemma}[thm]{Lemma}
\newtheorem{axiom}[thm]{Axiom}

\theoremstyle{definition}
\newtheorem{defn}[thm]{Definition}

\theoremstyle{definition}
\newtheorem{example}[thm]{Example}

\theoremstyle{definition}
\newtheorem{assumption}[thm]{Assumption}

\theoremstyle{remark}
\newtheorem{remark}[thm]{Remark}


\onehalfspace
\setlength{\parskip}{1ex}


%%%%%%%%%%%%%%%%%%%%%%%%%%%%%%%%%%%%%%%%%%%%%%%%%%%%%%%%%%%%%%%%%%%%%%%%%%%
%%%%%%%%%%%%%%%%%%%%%%%%%%%%%%%%%%%%%%%%%%%%%%%%%%%%%%%%%%%%%%%%%%%%%%%%%%%

\begin{document}

%%%%%%%%%%%%%%%%%%%%%%%%%%%%%%%%%%%%%%%%%%%%%%%%%%%%%%%%%%%%%%%%%%%%%%%%%%%
%%%%%%%%%%%%%%%%%%%%%%%%%%%%%%%%%%%%%%%%%%%%%%%%%%%%%%%%%%%%%%%%%%%%%%%%%%%

\title{Causal Inference in Python: A Vignette}
\author{Laurence Wong}
\maketitle

\section*{Installation}

CausalInference can be installed using \texttt{pip}, and will run provided the necessary dependencies are in place. On Ubuntu systems, the following commands should take care of all the essential steps if you are starting from scratch:
\begin{verbatim}
  $ sudo apt-get update
  $ sudo apt-get install python-pip python-numpy python-scipy
  $ sudo pip install causalinference
\end{verbatim}

\section*{Minimal Example}

The main object of interest in CausalInference is the class \texttt{CausalModel}. It takes as inputs three NumPy arrays: \texttt{Y}, an $N$-vector of observed outcomes; \texttt{D}, an $N$-vector of treatment status indicators; and \texttt{X}, an $N$-by-$K$ matrix of covariates. The following code snippet illustrates how the class can be invoked using random data simulated from NumPy's \texttt{random} module.

\begin{verbatim}
  >>> import numpy as np
  >>> from causalinference import *
  >>> Y = np.random.rand(1000)
  >>> D = np.random.randint(0, 2, 1000)
  >>> X = np.random.rand(1000, 3)
  >>> causal = CausalModel(Y, D, X)
\end{verbatim}

\section*{Class Attributes and Methods}

Once an instance of the class \texttt{CausalModel} has been created, it will contain a number of attributes and methods that are relevant for conducting a causal analysis. Tables \ref{tab.a} and \ref{tab.b} contain a brief description of these attributes and methods. \\

\begin{table}[h]
\begin{center}\begin{tabular}{ll}
Attribute & Description \\
\texttt{N, N\_c, N\_t, K} & Integers indicating sample sizes and number of covariates. \\
\texttt{covariates} & Dictionary-like object containing summary statistics for the \\
& covariate variables. \\
\texttt{pscore} & Dictionary-like object containing propensity score data, \\
& including estimated logistic regression coefficients, predicted \\
& propensity score, maximized log-likelihood, and the lists of the \\
& linear and quadratic terms that are included in the regression. \\
\texttt{cutoff} & Floating point number specifying the cutoff point for trimming \\
& on propensity score.\\
\texttt{blocks} & Either an integer indicating the number of equal-sized blocks to \\
& stratify the sample into, or a list of ascending numbers specifying \\
& the boundaries of the strata. \\
\texttt{strata} & List-like object containing the list of stratified propensity bins. \\
\texttt{est} & Dictionary-like object containing treatment effect estimates for \\
& each estimator used.
\end{tabular}\end{center}
\caption{Attributes of the class CausalModel. Invoking \texttt{print} on any of the dictionary- or list-like attribute above yields customized summary tables. Note that some attributes are only created after the relevant methods have been called.}  \label{tab.a}
\end{table}


\begin{table}[h]
\begin{center}\begin{tabular}{ll}
Method & Description \\
\texttt{restart} & Reinitializes data to original inputs, and drop any \\
& estimated results. \\
\texttt{propensity} & Estimates via logit the propensity score using specified \\
& linear and quadratic terms. \\
\texttt{propensity\_s} & Estimates via logit the propensity score using the \\
& covariate selection algorithm of Imbens and Rubin (2015). \\
\texttt{trim} & Trims data based on propensity score using the threshold \\
& specified by the attribute \texttt{cutoff}. \\
\texttt{trim\_s} & Trims data based on propensity score using the cutoff \\
& selected by the procedure of Crump, Hotz, Imbens, \\
& and Mitnik (2008). \\
\texttt{stratify} & Stratifies the sample based on propensity score as \\
& specified by the attribute \texttt{blocks}. \\
\texttt{stratify\_s} & Stratifies the sample based on propensity score \\
& using the bin selection procedure suggested by \\
& Imbens and Rubin (2015). \\
\texttt{blocking} & Estimates average treatment effects using regression \\
& within blocks. \\
\texttt{matching} & Estimates average treatment effects using matching \\
& with replacement. \\
\texttt{weighting} & Estimates average treatment effects using the \\
& Horvitz-Thompson weighting estimator modified to \\
& incorporate covariates. \\
\texttt{ols} & Estimates average treatment effects using least squares.
\end{tabular}\end{center}
\caption{Methods of the class CausalModel. Invoke \texttt{help} on any of the above methods for more detailed documentation.}  \label{tab.b}
\end{table}

%%%%%%%%%%%%%%%%%%%%%%%%%%%%%%%%%%%%%%%%%%%%%%%%%%%%%%%%%%%%%%%%%%%%%%%%%%%
%%%%%%%%%%%%%%%%%%%%%%%%%%%%%%%%%%%%%%%%%%%%%%%%%%%%%%%%%%%%%%%%%%%%%%%%%%%

\end{document}
